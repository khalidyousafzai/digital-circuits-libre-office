\باب{بنیادی حقائق}

\حصہ{اعشاری نظامِ اعداد}

عام زندگی میں اعشاری نظام اعداد  \index{ اعشاری نظام اعداد} استعمال ہوتا ہے جو 0-9  کے ہندسوں پر مبنی ہے ۔
مساوات \حوالہ{مساوات۔پانچ۔تین۔آٹھ۔اعشاری} میں 538 کا عدد لکھتے ہوئے اس کی دائیں
  جانب نیچے کر کے چھوٹی لکھائی میں  10 لکھا گیا ہے۔یہ اس بات کی یاد دہانی کراتا ہے کہ ہم اعشاری نظام استعمال
 کر رہے ہیں جس میں  10مختلف ہندسے پائے جاتے ہیں یعنی یہ نظام اساس دس \حاشیہب{base 10} پر مبنی ہے۔
اس کتاب میں چونکہ مختلف نظامِ اعداد استعمال ہوں گے لہٰذا اعداد کے ساتھ ان کا اساس اسی طرح لکھا جائے گا۔

\begin{align}
 538_{10}   \label{مساوات۔پانچ۔تین۔آٹھ۔اعشاری}
\end{align}

اس نظام میں دائیں جانب سے پہلا ہندسہ اکائی وزن رکھتا ہے، دوسرا دہائی، تیسرا سینکڑا وغیرہ
۔یوں مساوات  \حوالہ{مساوات۔پانچ۔تین۔آٹھ۔اعشاری} میں  دئے گئے ہندسہ میں دائیں جانب سے دیکھتے
 ہوئے  \(8\) کا مطلب \(8_{10}\) ہی ہے جبکہ  \(3\) کا مطلب  \(30_{10}\) اور  \(5\) کا مطلب  \(500_{10}\) ہے یعنی

\begin{align}
 538_{10}=\left(5 \times 10^{2}\right )_{10} + \left( 3 \times 10^1 \right)_{10} + \left(8 \times 10^0 \right)_{10}  \label{equWeightDecimal}
\end{align}

اعشاری نظام میں اعداد لکھتے ہوئے دائیں جانب سے پہلے ہندسے
 کا وزن  \(10^0\)ہوتا ہے  \حاشیہد{\(10^0=1\)} دوسرے ہندسے کا \(10^1\)، تیسرے کا \(10^2\) وغیرہ۔

اعشاری نظام میں گنتی  \(0_{10}\) سے شروع ہوتی ہے اور بتدریج
 بڑھتے ہوئے \(9_{10}\) تک پہنچتی ہے۔ اس دوران دہائی، سینکڑا وغیرہ کے مقام پر
 صفر رہتا ہے اور انہیں عام طور نہیں لکھا  جاتا۔مساوات \حوالہ{مساوات۔بائیں۔جانب۔صفر۔نہیں۔لکھتے} میں
 سات کو تین مختلف طریقوں سے لکھا گیا ہے۔عام زندگی میں سات کو مساوات کی پہلی طرز پر لکھا جاتا ہے۔

\begin{align} 
 7_{10} & \notag \\
07_{10} &  \notag \\
007_{10} &  \label{مساوات۔بائیں۔جانب۔صفر۔نہیں۔لکھتے}
\end{align}

اعشاری نظام میں گنتی \(0\) سے شروع ہوتی ہے۔ \(9\) تک پہنچنے کے
 بعد دہائی یعنی \(10^1\) وزن رکھنے والی مقام پر  \(0\) کی بجائے  \(1\) لکھا جاتا
 ہے اور اکائی یعنی \(10^0\) وزن رکھنے والی مقام پر دوبارہ \(0\) سے \(9\) کی جانب گنتی شروع ہوتی ہے۔

اگر آپ کو اس پیراگراف کی سمجھ نہیں آئی تو اسے دوبارہ پڑھیں۔اس میں سادہ گنتی کی وضاحت کی گئی ہے۔


اعشاری نظام میں اگر اعداد کو ایک ہندسے تک محدود کر دیا جائے
 تو اس میں  \(0_{10}\) سے  \(9_{10}\) تک گنتی ممکن ہو گی۔اگر اعداد
 کو دو ہندسوں تک محدود کر دیا جائے تو \(00_{10}\) سے  \(99_{10}\) تک گنتی ممکن
 ہو گی، تین ہندسوں کا عدد  \(000_{10}\) سے \(999_{10}\) تک گن سکتا ہے وغیرہ۔

\حصہ{ہشتم نظامِ اعداد}

	ہشتم نظامِ اعداد  \(0-7\) کے ہندسوں پر مبنی ہے۔یوں اس  نظام میں آٹھ ہندسے ہیں اور یوں یہ اساس  \(8\)  کا نظام ہے۔اس نظام میں اعداد لکھتے ہوئے دائیں
 جانب سے پہلے ہندسے  کا وزن  \(8^0\) ہوتا ہے \حاشیہد{ \(8^0=1\) } دوسرے ہندسے کا \(8^1\) ، تیسرے کا  \(8^2\) وغیرہ۔

\begin{align}
 537_8 &=\left( 5 \times 8^2 \right)_{10} + \left(3 \times 8^1 \right)_{10} +\left(7 \times 8^0 \right)_{10} \notag \\
&=\left( 5 \times 64 \right)_{10} +\left( 3 \times 8 \right)_{10} +\left( 7 \times 1\right)_{10} \notag \\
&= \left( 320 +24 + 7\right)_{10} \notag \\
&= 351_{10}    \label{equOctalToDecimal}
\end{align}

مساوات \حوالہ{equOctalToDecimal} میں ہشتم نظام میں دئے گئے عدد کو اعشاری نظام میں تبدیل کرنا
 دکھایا گیا ہے۔ہشتم عدد کی دائیں جانب نیچے کر کے چھوٹی لکھائی میں \(8\) اس بات کی یاد دہانی کرتا ہے کہ یہ عدد ہشتم نظام میں لکھا گیا ہے۔

اس نظام میں گنتی \(0\) سے شروع ہوتی ہے۔\(7\) تک پہنچنے کے  بعد \(8^1\)  وزن رکھنے والی مقام پر \(0\) کی بجائے \(1\) لکھا
 جاتا ہے اور \(8^0\) وزن رکھنے والی مقام پر دوبارہ \(0\) سے  \(7\) کی جانب گنتی شروع ہوتی ہے۔

\حصہ{دہری نظامِ اعداد}

مائکرو کنٹرولر کی دنیا میں دہری نظام استعمال ہوتا ہے۔دہری نظامِ اعداد  \(0-1\) کے ہندسوں پر مبنی ہے۔یوں اس نظام
 میں دو ہندسے ہیں اور یوں اس کا اساس  \(2\) ہے۔اس نظام میں اعداد لکھتے ہوئے دائیں جانب سے پہلے ہندسے
 کا وزن  \(2^0\) ہوتا ہے \حاشیہد{\(2^0=1\)} دوسرے  ہندسے کا  \(2^1\) ، تیسرے کا  \(2^2\) وغیرہ۔


\begin{align}
 1011_{2} &=\left(1 \times 2^3 \right)_{10} + \left(0 \times 2^2 \right)_{10} +\left(1 \times 2^1 \right)_{10} +\left( 1 \times 2^0\right)_{10} \notag \\
&=\left(1 \times 8\right)_{10} +\left(0 \times 4\right)_{10} +\left( 1 \times 2 \right)_{10}+\left(1 \times 1 \right)_{10} \notag \\
&=\left( 8+0+2+1 \right)_{10} \notag \\
&=11_{10}                                    \label{equBinaryToDecimal}
\end{align}

مساوات  \حوالہ{equBinaryToDecimal} میں دہری
 نظام میں دئے گئے عدد کو اعشاری نظام میں تبدیل کرنا دکھایا گیا ہے۔ دہری عدد کی دائیں جانب
 نیچے کر کے چھوٹی لکھائی میں  \(2\) اس بات کی یاد دہانی کرتا ہے کہ یہ عدد دہری نظام میں لکھا گیا ہے۔


اس نظام میں گنتی  \(0\) سے شروع ہوتی ہے۔ \(1\)  تک پہنچنے کے بعد  \(2^1\) وزن رکھنے  والی مقام پر \(0\) کی
 بجائے  \(1\) لکھا جاتا ہے اور  \(2^0\) وزن رکھنے والی مقام پر دوبارہ  \(0\) سے  \(1\) کی جانب
 گنتی شروع ہوتی ہے۔یہ مساوات \حوالہ{equDecimalBinaryOctalHex} میں دکھایا گیا ہے۔


اس مساوات سے یہ بھی واضح ہے کہ  دہری نظام میں چار ہندسوں کا عدد  \(0000_2\) سے  \(1111_2\) تک
 کی گنتی کے لئے استعمال ہو سکتا ہے۔اگر اس سے بڑا عدد لکھنا ہو تو چار
 سے زیادہ ہندسے استعمال کرنا ضروری ہو گا۔مائکرو کنٹرولر آٹھ ہندسوں کے اعداد استعمال کرتا ہے۔
 آٹھ ہندسوں میں  \(00000000_2\) سے \(11111111_2\) تک کے اعداد ظاہر کئے جا سکتے ہیں۔

\begin{align}  \label{equDecimalBinaryOctalHex}
 00_{10} & = 00_8  = 0000_2 = 0_{16}  \notag \\
 01_{10} & = 01_8  = 0001_2 = 1_{16} \notag \\
 02_{10} & = 02_8  = 0010_2 = 2_{16} \notag \\
 03_{10} & = 03_8  = 0011_2 = 3_{16} \notag \\
 04_{10} & = 04_8  = 0100_2 = 4_{16} \notag \\
 05_{10} & = 05_8  = 0101_2 = 5_{16} \notag \\
 06_{10} & = 06_8  = 0110_2 = 6_{16} \notag \\
 07_{10} & = 07_8  = 0111_2 = 7_{16} \notag \\
 08_{10} & = 10_8  = 1000_2 = 8_{16} \notag \\
 09_{10} & = 11_8  = 1001_2 = 9_{16} \notag \\
 10_{10} & = 12_8  = 1010_2 = A_{16} \notag \\
 11_{10} & = 13_8  = 1011_2 = B_{16} \notag \\
 12_{10} & = 14_8  = 1100_2 = C_{16} \notag \\
 13_{10} & = 15_8  = 1101_2 = D_{16} \notag \\
 14_{10} & = 16_8  = 1110_2 = E_{16} \notag \\
 15_{10} & = 17_8  = 1111_2 = F_{16}
\end{align}

\حصہ{اساس سولہ کا نظامِ اعداد}

اساس سولہ کے نظامِ اعداد میں سولہ علامتیں ہیں۔ان میں پہلی دس علامتیں  \(0-9\) ہیں اور بقایا بڑی لکھائی میں انگریزی حروفِ
 تہجی کے پہلے چہ حروف یعنی  \(ABCDEF\) ہیں۔ ان میں  \(F_{16}\) سب سے بڑا ہندسہ ہے اور یہ پندرہ کو واضح کرتا ہے۔

اس نظام میں اعداد لکھتے ہوئے دائیں جانب سے پہلے ہندسے
 کا وزن  \(16^0\) ہوتا ہے \حاشیہد{  \(16^0=1\)  } دوسرے ہندسے کا \(16^1\) ، تیسرے کا  \(16^2\) وغیرہ۔

\begin{align}  \label{equHexToDecimal}
 3AC_{16} &=\left(3 \times 16^2 \right)_{10}+\left( 10 \times 16^1  \right)_{10} + \left( 12 \times 16^0 \right)_{10} \notag \\
&=\left(3 \times 256\right)_{10} +\left( 10 \times 16 \right)_{10} +\left( 12 \times 1 \right)_{10} \notag \\
&=940_{10}
\end{align}

مساوات \حوالہ{equHexToDecimal} میں اساس سولہ کے نظام میں دئے گئے عدد کو اعشاری نظام میں تبدیل کرنا دکھایا گیا ہے
 اور مساوات \حوالہ{equDecimalBinaryOctalHex} میں دائیں جانب اس کی گنتی دکھائی گئی ہے۔

\حصہ{اساس دو سے اساس آٹھ اور اساس سولہ حاصل کرنا}

مساوات \حوالہ{equBinaryToOctal} میں  \(1101100_2\)  کا دہری عدد 
بائیں جانب دیا گیا ہے۔اس دہری عدد کو اساس آٹھ میں لکھنے کی خاطر پہلے اس کو دائیں 
جانب سے تین تین کے گروہ میں لکھیں۔بائیں اگر تین کا گروہ پورا نہ ہو تو اک کی بائیں جانب صفر لگا کر اسے پورا کریں۔
اب مساوات \حوالہ{equDecimalBinaryOctalHex} کی مدد سے ان تین تین کے گروہ کی جگہ ان کا مثاوی اساس آٹھ 
کا ہندسہ لکھیں۔مساوات \حوالہ{equBinaryToOctal} میں یوں دائیں جانب سے  \(100_2\) کی جگہ  \(4_8\) لکھا
 گیا ہے، \(101_2\)  کی جگہ  \(5_8\) اور  \(001_2\) کی جگہ  \(1_8\) لکھا گیا ہے۔
یوں یہ عدد اساس آٹھ میں  \(158_8\) لکھا جائے گا۔


\begin{align}  \label{equBinaryToOctal}
 1101100_2 &= \left(001 \phantom{0} 101 \phantom{0} 100 \right)_{2} \notag \\
&=\left(1 \phantom{010} 5 \phantom{010} 4 \right)_8 \notag \\
&=154_8
\end{align}

دہری عدد کو اساس سولہ میں لکھنے کی خاطر دہری عدد کو دائیں جانب سے
  چار چار کے گروہ میں لکھیں۔اگر چار کا گروہ پورا نہ ہو تو اس کے  بائیں جانب صفر لگا کر اسے پورا کریں۔اب
 مساوات \حوالہ{equDecimalBinaryOctalHex} کی مدد سے ان چار چار کے گروہ کی جگہ ان کا مثاوی اساس سولہ کا ہندسہ لکھیں۔
مساوات \حوالہ{equBinaryToHex} میں یوں دائیں جانب سے \(1100_2\) کی جگہ  \(C_{16}\) لکھا گیا ہے
 اور  \(0110_2\) کی جگہ  \(6_{16}\) لکھا گیا ہے۔یوں یہ عدد اساس سولہ میں  \(6C_{16}\) لکھا جائے گا۔

\begin{align} \label{equBinaryToHex}
 1101100_2 &= \left( 0110 \phantom{0} 1100\right)_2 \notag \\
&=\left(  6 \phantom{1100} C \right)_{16} \notag \\
&=6C_{16}
\end{align}

انہیں طریقوں کو الٹ استعمال کرتے ہوئے اساس آٹھ اور اساس سولہ کے
 اعداد با آسانی دہری عدد کے طور لکھے جاتے ہیں۔مساوات \حوالہ{equOctalToBinary} میں اساس آٹھ
 اور مساوات \حوالہ{equHexToBinary} میں اساس سولہ کو دہری عدد کی شکل میں لکھنا دکھایا گیا ہے۔

\begin{align}  \label{equOctalToBinary}
 372_8 &=\left(3 \phantom{110} 7 \phantom{110} 2 \right)_8 \notag \\
&=\left( 011 \phantom{0} 111 \phantom{0} 010 \right)_{2} \notag \\
&=11111010_2
\end{align}

\begin{align} \label{equHexToBinary}
 9A2F_{16} &=\left( 9 \phantom{0010} A \phantom{0100}  2 \phantom{1010}  F \right)_{16} \notag \\
&=\left(1001 \phantom{0} 1010 \phantom{0}  0010  \phantom{0} 1111 \right)_2 \notag \\
&=1001101000101111_2
\end{align}

مساوات \حوالہ{equOctalToBinary} اور \حوالہ{equHexToBinary} کی آخری لکیروں میں دہری اعداد کو دیکھتے ہوئے بہت جلد انسان اکتا جاتا ہے
 البتہ انہیں مساوات میں جہاں ان اعداد کو گروہ کی شکل میں لکھا گیا ہے وہاں انہیں سمجھنا ممکن ہے۔

دہری اعداد کو جب  آٹھ کے گروہ میں لکھا جائے  تو اسے ایک ہشتم دہری
 عدد یا ایک بائٹ کہتے ہیں۔بائٹ کو  عموماً دو چار چار دہری اعداد کی گروہ میں لکھا جاتا ہے۔یوں
مساوات \حوالہ{equHexToBinary} میں دو بائٹ ہیں۔اسی مساوات کو الٹ چلاتے ہوئے  یہ واضح ہے کہ ہشتم 
دہری عدد کو چار-چار دہری اعداد کی  گروہ میں لکھ کر انہیں جلد اساس سولہ میں لکھا جا سکتا ہے۔

\حصہ{اساسی تکملہ}

کسی بھی اساسی نظام میں اگر ایک ہندسے پر مبنی عدد کو اساس سے منفی کیا جائے تو حاصل
 جواب اس عدد کا اساسی تکملہ \حاشیہب{radix complement}  کہلاتا
 ہے۔یوں اس عدد اور عدد کے اساسی تکملہ کا مجموعہ اساس کے
 برابر ہو گا۔اعشاری نظام کی مثال لیتے ہوئے 3 کا اساسی تکملہ 7  ہے اور 7  کا اساسی
 تکملہ 3 ہے، 5 کا اسای تکملہ 5 ہے، 9  کا اساسی تکملہ 1 ہے وغیرہ۔ان مثالوں سے
 یہ واضح ہے کہ کسی بھی عدد کے اساسی تکملہ کا اساسی تکملہ وہی عدد ازخود ہوتا ہے۔

اسی تصور کو آگے بڑھاتے ہوئے ایک سے
 زیادہ ہندسوں پر مبنی عدد کے لئے اساسی تکملہ یوں بیان کیا جاتا ہے۔اساس r کے اعدادی
 نظام میں عدد N جس کے n ہندسے ہوں کا اساسی تکملہ سے مراد عدد \(r^n-N\) ہے۔

اعشاری نظام میں \(10^n\)  ایک ایسا عدد بنتا ہے جس میں سب سے زیادہ وزن والا ہندسہ ایک ہوتا ہے اور
 اس کے بعد \(n\) صفر ہوتے ہیں۔لہٰذا \(10^3=1000\) اور \(10^7=10000000\) ہے۔


اعشاری نظام کا اساس 10 ہے۔اس نظام میں ایک 
عدد N جس کے n ہندسے ہوں کے اساسی تکملہ سے مراد \(10^n-N\) ہے۔
یوں \(N=5391\)  میں چار ہندسے ہیں یعنی \(n=4\)  ہے لہٰذا اس کا اساسی تکملہ

\begin{align}
10^4 - 5391 &= 10000-5391 =4609
\end{align}
ہے۔اسی طرح ایک عدد \(320753\) جس میں \(6\) ہندسے ہیں کا اساسی تکملہ

\begin{align}
10^6 - 320753 &= 1000000-320753 =679247
\end{align}
ہے۔ایک آخری مثال لیتے ہیں۔ 679247 کا اساسی تکملہ
\begin{align}
10^6 - 679247 &= 1000000-679247 =320753
\end{align}

ہے۔کسی بھی عدد کا اساسی تکملہ کا اساسی
تکملہ وہی عدد ازخود ہوتا ہے۔اس بات کو یوں ثابت کر سکتے ہیں کہ \(N\) کا اساسی
 تکملہ \(r^n-N\) ہے اور \(r^n-N\) کا اساسی تکملہ \(r^n-(r^n-N)\)  یعنی \(N\) ہے۔

دہری نظام  اساس 2 ہے لہٰذا n ہندسوں پر مبنی دہرے عدد N کا اساسی تکملہ \(2^n-N\)  ہو گا۔دہری نظام 
میں \(2^n\)  ایک ایسا عدد بنتا ہے جس میں سب سے زیادہ وزن والا ہندسہ ایک ہوتا ہے اور اس
 کے بعد \(n\) صفر ہوتے ہیں۔لہٰذا \(2^4=10000_2\) اور \(2^8=100000000_2\) ہوتا ہے۔
اس طرح  \(1011_2\) اور \(1001_2\) کے اساسی تکملہ \(0011_2\) اور \(0111_2\)  ہیں۔

اساس دس کے اساسی تکملہ کو عام طور 10 کا تکملہ \حاشیہب{10's complement} کہتے ہیں۔
اسی طرح اساس دو کے تکملہ کو 2 کا تکملہ \حاشیہب{2's complement} کہتے ہیں۔

\حصہ{اساس منفی ایک کا تکملہ}
اساس r کے نظامِ اعداد میں ایک ہندسے پر مبنی
 عدد کے اساس منفی ایک کے تکملہ \حاشیہب{radix-1 complement} سے
 مراد  \(r^n-1-N\) ہے۔اعشاری نظامِ گنتی میں اساس منفی ایک کے تکملہ
 کو عموما  9 کا تکملہ کہتے ہیں اور دہری نظامِ گنتی میں اسے عموما  1 کا تکملہ کہتے ہیں۔

اعشاری نظام میں \(376\) اور \(7852319\) کے 9 کے تکملہ مندرجہ ذیل ہیں۔
\begin{align}
10^3 -1- 376 &= 1000-1-376 =999-376 =623 \\
10^7-1-7852319 &=10000000-1-7852319=9999999-7852319=2147680
\end{align}

اعشاری نظام میں \(10^n-1\) ایک ایسا عدد بنتا ہے جس میں n ہندسے ہوتے ہیں
اور ہر ہندسہ 9 ہوتا ہے۔یعنی \(10^3-1=999\) اور \(10^5-1=99999\) ہوتا
 ہے۔یوں \(376\) کا اساس منفی ایک کا تکمل اس طرح حاصل کیا جاتا ہے۔


\begin{LTR}
\begin{tabular}{cccc}
& 9&9 &9 \\
- & 3&7&6 \\
\hline
& 6&2 &3 \\
\end{tabular}
\end{LTR}

دہری نظام میں \(2^n-1\) ایک ایسا عدد بنتا ہے جس میں n ہندسے ہوتے ہیں اور ہر
 ہندسہ 1 ہوتا ہے۔یعنی \(2^3-1=111_2\) اور \(2^8-1=11111111_2\) ہوتا ہے۔

اس طرح دہری نظامِ گنتی میں \(101110_2\)  اور \(1001_2\) کا 1 کا تکملہ 
\begin{align}
2^6 -1- 101110 &= 1000000-1-101110 =111111-101110 =010001 \\
2^4-1-1001 &=10000-1-1001=1111-1001=0110
\end{align}

ان دو مثالوں میں ایک اہم بات سامنے آتی ہے۔کسی بھی دہرے عدد کا 1 کا تکملہ یوں حاصل کیا
 جا سکتا ہے کہ اس عدد میں ہر 0 کی جگہ 1 لکھ لیا جائے اور ہر 1 کی جگہ 0 لکھ لیا جائے۔

\حصہ{تکملہ کی مدد سے منفی}
عام زندگی میں قلم سے منفی کرنا چھوٹی کلاسوں میں سکھایا جاتا ہے۔الیکٹرانکس میں تکملہ
 کی مدد سے دو اعداد منفی کئے جاتے ہیں۔اساسی تکملہ سے \(M-N\) مندرجی
 ذیل قدموں پر چلنے سے حاصل کیا جاتا ہے۔اگر ان دو اعداد میں ہندسوں کی تعداد
 برابر نہ ہو تو کم ہندسوں پر مبنی عدد کے بائیں جانب اضافی صفریں لگا کر اس عدد میں ہندسوں 
کی تعداد زیادہ ہندسوں پر مبنی عدد کے برابر کریں۔\todo{turn this all to libre office}

\begin{itemize}
\item {جو عدد منفی کیا جا رہا ہو اس کے اساسی تکملہ \(r^n-N\)  اور دوسرے عدد کا
 مجموعہ لیں یعنی \(M+(r^n-N)\)}
\item {اگر M کی مقدار  N کی مقدار سے زیادہ ہو تو یہ مجموعہ \(n+1\) ہندسوں پر مشتمل ہو گا۔اس مجموعہ کے
 دایاں ہندسہ
  کی قیمت ہر
 صورت 1 ہو گی۔
اس دائیں 1 کو
 نظر انداز 
کر کے بقایا \(n\) ہندسوں پر مبنی عدد کو لیں۔یہی جواب ہو گا}
\item {اگر M کی مقدار N کی مقدار سے کم ہو تب یہ مجموعہ \(n\) ہندسوں پر ہی مبنی ہوگا۔اس صورت میں مجموعہ کا اساسی
 تکملہ لیں۔یہ جواب ہوگا}
\end{itemize}

یہ دونوں صورتیں مثالوں سے واضح ہوں گی۔

سوال:
دس کے تکملہ کی مدد سے \(7852-974\) حاصل کریں۔

جواب:
یہاں بڑا عدد 7852 چار ہندسوں پر مبنی ہے لہٰذا 974 کے 10 کا تکملہ لیتے وقت n کو چار تصور کیا جائے گا۔یوں 974 کے 10 کا تکملہ

