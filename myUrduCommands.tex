%this file describes urdu commands for the commonly used english latex commands

%chapter, section etc
\newcommand{\باب}[1]{\chapter{#1}}                                      %defining commonly used commands
\newcommand{\حصہ}[1]{\section{#1}}
\newcommand{\جزحصہ}[1]{\subsection{#1}}
\newcommand{\جزجزحصہ}[1]{\subsubsection{#1}}

%english text in urdu mode
\newcommand{\تحریرانگریزی}[1]{\textenglish{#1}}	% english text in urdu mode

%end commands cannot be redefined and as such these two are not usable
%\newcommand{\ابتدا}[1]{\begin{#1}}
%\newcommand{\انتہا}[1]{\end{#1}}

%include and input directives for adding external files into the main document 
%\newcommand{\بشمول}[1]{\includeonly{#1}}
%\newcommand{\شامل}[1]{\include{#1}}
%\newcommand{\داخل}[1]{\input{#1}}

%to use extra latex packages
%\newcommand{\استعمال}[1]{\usepackage{#1}}

%footnotes and indexes
%\newcommand{\حاشیہب}[1]{{\raggedright{\footnote{\textenglish{#1}}}} }    %footnote to the left hand side
%\newcommand{\حاشیہد}[1]{{\raggedleft{\footnote{#1}}}}
%\newcommand{\حاشیہط}[1]{\marginpar{#1}}										%margin notes

%references and labels
\newcommand{\شناخت}[1]{\label{#1}}
\newcommand{\حوالہ}[1]{\ref{#1}}

%itemize, bullets and numbered items   
\newcommand{\اشیاء}{itemize}                               %used in   \begin{itemize}
\newcommand{\شے}[1]{\item {#1}}			%used in    \item

%maths commands
\newcommand{\عددی}[1]{\ensuremath {#1}}  %this works both in-line and in math mode
\newcommand{\سیدھ}{align}					%used in    \begin{align}
%\newcommand{\فاصلہ}[1]{\hspace{#1}}
%\newcommand{\حکم}[2]{\hspace{1in} {#1} \hspace{0.5in} {#2}}

\newcommand{\بائیں}[1]{ \left #1}	%all from left to right bracket must be enclosed in \عددی{} command
\newcommand{\دائیں}[1]{ \right #1} %all from left to right bracket must be enclosed in \عددی{} command

\newcommand{\ضرب}{\time}					%multiplication symbol

